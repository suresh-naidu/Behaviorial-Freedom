\documentclass[10pt,a4paper]{article}
\usepackage[T1]{fontenc}
%\usepackage{cmbright}
\usepackage{colortbl}

\usepackage{dcolumn}
\newcolumntype{.}{D{.}{.}{-1}}
\newcolumntype{d}[1]{D{.}{.}{#1}}

\usepackage{amsmath}
\usepackage{amsfonts}
\usepackage{amssymb}

\usepackage{graphicx}

\usepackage{setspace}

\usepackage{rotating}
\usepackage{rotfloat}

\title{A Behavioral Theory of Freedom}  
\author{Suresh Naidu and Christopher Muller}
\begin{document}
\maketitle

\begin{abstract}
\end{abstract}

\setcounter{page}{0}
\thispagestyle{empty}

\renewcommand{\baselinestretch}{1.4}\large\normalsize

\section*{Introduction}

\noindent Classical liberalism defines freedom as some variant of an atomistic individual's untrammeled ability to choose. Critics object that this hypothetical individual has no empirical equivalent. It is impossible to know what choice in the absence of social constraints would look like because we never observe individuals completely severed from a web of social obligations. We argue that both the liberal definition of freedom and this particular criticism of it are wrong. Liberalism's critics are correct that freedom cannot be understood in isolation from social connections and constraints. But this is not because liberalism's hypothetical atomistic individuals cannot be found. It is because when we do find them, we confront individuals who fundamentally lack the cognitive capacity to choose. 

freedom: an empirical view. The relaxation of constraint view.

3 cases
prisons, indentures, sex workers.

solitary confinement (Guenther 2013): eliminate ties, lose capacity to reason
ties to kin the binding constaint that creates the capacity for future freedom

Liberalism



Economics
 
 The only perfectly free market is a slave market. But the macrofoundations of actual slave markets 
 (natal alienation, severance of social ties) undermine the hypothetical microfoundations of free markets. 
 ``You don't get somebody who is capable of making decisions when they are socially monocropped.''
 
 socially disembedded
 
 Social death as the original tunneling: Slave's obsession with master (Patterson 1982); 
 relationship of the deprived to what they do not have (Mullainathan and Shafir 2013); prisoner 
 in solitary's obsession with the guard (Grassian; Gawande)
 
 The importance of multiple redundant sources of social status: commensuration (as in a slave, a worker, 
 abstract labor) reduces status to one dimension (Espeland and Stevens). Tunneling leads to the incapacity 
 to make decisions. Single dimension of status leads people to seek other dimensions (networks of 
 of crime; gangs) 
 
 We can observe the effects of social embeddedness negatively. 
 
 Graeber: fungibility of persons
 
 What is the point of this paper? It is Suresh's answer to Daron's question: 
 in what sense are these coerced laborers not free?
 
 Drunken people dragooned onto boats. You were being really stupid at time T. 
 
\end{document}


x

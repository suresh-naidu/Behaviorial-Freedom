\documentclass[10pt,a4paper]{article}
\usepackage[T1]{fontenc}
%\usepackage{cmbright}
\usepackage{colortbl}

\usepackage{dcolumn}
\newcolumntype{.}{D{.}{.}{-1}}
\newcolumntype{d}[1]{D{.}{.}{#1}}

\usepackage{amsmath}
\usepackage{amsfonts}
\usepackage{amssymb}

\usepackage{graphicx}

\usepackage{setspace}

\usepackage{rotating}
\usepackage{rotfloat}

\title{A Behavioral Theory of Freedom}  
\author{Suresh Naidu and Christopher Muller}
\begin{document}
\maketitle

\begin{abstract}
\end{abstract}

\setcounter{page}{0}
\thispagestyle{empty}

\renewcommand{\baselinestretch}{1.4}\large\normalsize

\section*{Introduction}

\noindent Classical liberalism defines freedom as some variant of an atomistic individual's untrammeled ability to choose. Critics object that this hypothetical individual has no empirical equivalent. It is impossible to know what choice in the absence of social constraints would look like because we never observe individuals completely severed from a web of social obligations. We argue that both the liberal definition of freedom and this particular criticism of it are wrong. Liberalism's critics are correct that freedom cannot be understood in isolation from social connections and constraints. But this is not because liberalism's hypothetical atomistic individuals cannot be found. It is because when we do find them, we confront individuals who fundamentally lack the cognitive capacity to choose. 

freedom: an empirical view. The relaxation of constraint view.

3 cases
prisons, indentures, sex workers.
 
\end{document}


x

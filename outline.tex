\documentclass[10pt,a4paper]{article}
\usepackage[T1]{fontenc}
%\usepackage{cmbright}
\usepackage{colortbl}

\usepackage{dcolumn}
\newcolumntype{.}{D{.}{.}{-1}}
\newcolumntype{d}[1]{D{.}{.}{#1}}

\usepackage{amsmath}
\usepackage{amsfonts}
\usepackage{amssymb}

\usepackage{graphicx}

\usepackage{setspace}

\usepackage{rotating}
\usepackage{rotfloat}

\title{A Behavioral Theory of Freedom}  
\author{Suresh Naidu and Christopher Muller}
\begin{document}
\maketitle

\begin{abstract}
\end{abstract}

\setcounter{page}{0}
\thispagestyle{empty}

\renewcommand{\baselinestretch}{1.4}\large\normalsize

\section*{Introduction}

a self has a utility function over consumption and savings, $V^i(a1,a2)$. Follow piketty and saez and have taste shocks.
subject to: a1_t+a2_t=f(a2_t-1)

special case for capital taxV=c^alpha_i s^(1-alpha_i)- l^epsilon.
c+s= wl 
In steady-state optimum policy solves \int V^i dF(i)

\noindent Classical liberalism defines freedom as some variant of an atomistic individual's untrammeled ability to choose. Critics object that this hypothetical individual has no empirical equivalent. It is impossible to know what choice in the absence of social constraints would look like because we never observe individuals completely severed from a web of social obligations. We argue that both the liberal definition of freedom and this particular criticism of it are wrong. Liberalism's critics are correct that freedom cannot be understood in isolation from social connections and constraints. But this is not because liberalism's hypothetical atomistic individuals cannot be found. It is because when we do find them, we confront individuals who fundamentally lack the cognitive capacity to choose. 

One of the broad themes of Hegel's philosophy of right is the contradiction between two types of "free will".
One is "free" as in unanchored, without any content of its own, and able to realize itself as any particular content.
The other is "free" in the sense of being able to realize a given content (absence of constraint).

What we aim to do in this paper is draw some connections between sociology, social psychology, and behavioral economics 
in expanding on the first notion of freedom. 

Behavioral economics hasn't recognized itself in hegel's distinction, but this might prove fruitful in understanding what the empirical
understanding of hman behavior and psychology actually do for public policy. What is the goal of the "state" when it has a role in constituting subjects
as well as incentivizing them?

Capacities vs Hegel? Why capabilities is a silly approach. Works only for those things that are "bare life"?

Can we operationalize this with generalized social welfare weights? If we want to maximize freedom that both creates the
subject and lets it realize its objectives how would we do that? What is the policy consequence of this view of freedom?

JALYs...the capability metric of the carceral society


freedom: an empirical view. The relaxation of constraint view.

``Confronted with the master's outrageous effort to deny him all dignity, the slave even more than the master came 
to know and to desire passionately this very attribute'' (Patterson 1982: 100)

behavioral contradictions of liberalism.

Indentures as a key problem for liberalism. Because you say yes at time t, but behavioral economics now makes you time inconsistent
so it may be a failure.

Need institutions and social structure to enable cognitively capable liberal agents.

3 cases
prisons, indentures, sex workers.

solitary confinement (Guenther 2013): eliminate ties, lose capacity to reason
ties to kin the binding constaint that creates the capacity for future freedom

Liberalism



Economics
 
 The only perfectly free market is a slave market. But the macrofoundations of actual slave markets 
 (natal alienation, severance of social ties) undermine the hypothetical microfoundations of free markets. 
 ``You don't get somebody who is capable of making decisions when they are socially monocropped.''
 
 socially disembedded
 
 Social death as the original tunneling: Slave's obsession with master (Patterson 1982); 
 relationship of the deprived to what they do not have (Mullainathan and Shafir 2013); prisoner 
 in solitary's obsession with the guard (Grassian; Gawande)
 
 The importance of multiple redundant sources of social status: commensuration (as in a slave, a worker, 
 abstract labor) reduces status to one dimension (Espeland and Stevens). Tunneling leads to the incapacity 
 to make decisions. Single dimension of status leads people to seek other dimensions (networks of 
 of crime; gangs) 
 
 The more you load status onto one dimension, the more people (a) tunnel, (b) seek out alternative
 sources of status (in family or in crime networks)
 
 
 Gawande:EEG readings of yugoslav prisoners. Cognitive effects of solitary confinement
 
 Mullainathan and Shafir loneliness makes you overfocus and choke. Optimal diversity of cognition.
 
 
 Slave narratives:who/what do you remember from slavery, cognitive biases of slaves. Ed baptist overinvestment in human capital for the master.
 
 Hegel philosophy of right.
 We can observe the effects of social embeddedness negatively. 
 
 Graeber: fungibility of persons
 
 What is the point of this paper? It is Suresh's answer to Daron's question: 
 in what sense are these coerced laborers not free?
 
 Drunken people dragooned onto boats. You were being really stupid at time T. 
 
 ``The most insidious effect of Roman slavery, however, is that through Roman law, 
 it has come to play havoc with our idea of human freedom. The meaning of the Roman 
 word libertas itself changed dramatically over time. As everywhere in the ancient world, 
 to be `free' meant, first and foremost, not to be a slave. Since slavery means above all 
 the annihilation of social ties and the ability to form them, freedom meant the capacity 
 to make and maintain moral commitments to others. The English word `free,' for instance, 
 is derived from a German root meaning 'friend,' since to be free meant to be able to 
 make friends, to keep promises, to live within a community of equals. This is why freed 
 slaves in Rome became citizens: to be free, by definition, meant to be anchored in a 
 civic community, with all rights and responsibilities that this entailed.'' (Graeber 203)
 
 ``Kings surround themselves with slaves for the same reason that they surround 
 themselves with eunuchs: because the slaves and criminals have no families or friends, 
 no possibilities of other loyalties...In other words, the king and slave are mirror images, 
 in that unlike normal human beings who are defined by their commitments to others, they are 
 defined only by relations of power. They are as close to perfectly isolated, alienated 
 beings as one can possibly become.'' (Graeber 209)
 
 Western: Leaving Prison as a Transition to Poverty: what is the best thing about being out?
 ``Family,'' ``Freedom''
 
\end{document}


x
